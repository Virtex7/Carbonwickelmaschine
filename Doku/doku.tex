\documentclass[12pt, a4paper, ngerman]{article}
\usepackage[utf8]{inputenc}
\usepackage[T1]{fontenc}
\usepackage[ngerman]{babel}
\date{\today}
\author{Philipp Hörauf, Toni Bartsch}
\title{DIY: Die Carbonwickelmaschine}

% Trennung von komischen Wörtern:


% Bülder
\usepackage{cleveref}
\usepackage{graphicx}

\begin{document}
\maketitle
\newpage
\tableofcontents
\newpage
\section{Einleitung}
Ziel des Projektes ist eine Carbonwickelmaschine zu entwickeln um zylindrische und konische Wickelkörper mit unterschiedlichen Rovings bewickeln zu können. Auf diese Weise soll es möglich sein extrem leichte, verschwindungssteife und korrosionsbeständig Rohre herzustellen die quasi immun gegen Wärmeausdehnung sind. Für diesen Zweck gibt es zwar bereits Maschinen, die aber entweder für deutlich größere Teile ausgelegt sind, oder die sehr teuer bzw. nicht mehr verfügbar sind. Beispiele für solche Maschinen sind: TODO

\section{Spezifikationen und Überblick}
Die geplanten Spezifikationen für die Maschine sind:

\begin{itemize}
    \item Maximale Wickellänge 1200mm
    \item Größter Durchmesser des Wickelkörpers 250mm
    \item Verarbeitung von Carbon- und Glasfasern
    \item Wickelvorgang PC-kontrolliert
    \item Möglichst kostengünstiger Aufbau
    \item Konstruktion und Fertigung der Maschine aus Aluminium und Holz
    \item 4 Achsen
\end{itemize}

\subsection{Elektronik}
\begin{itemize}
    \item 4 Schrittmotoren
    \item GRBL-Board
    \item Netzteil
    \item Endschalter
\end{itemize}


\subsection{Programm}
\begin{itemize}
    \item G-Code Generator mit graphischem Userinterface anhand von einstellbaren Parametern
    \item grbl Interface zum streamen des G-Codes und manuellem Verfahren
\end{itemize}

\section{Grundlagen}
Um die Funktionsweise und den Aufbau der Wickelmaschine besser verstehen zu können sollen als erstes einige wichtige Grundlagen zu den verwendeten Faserverbundwerkstoffen sowie dem Wickelvorgang erklärt werden. Die Wickelmaschine arbeitet grundsätzlich nach dem Prinzip, dass eine Form, z.B. ein Rohr mit einem Faserverbundwerkstoff umwickelt wird. Je nach Form und Vorbehandlung verbleibt die Form im fertigen Teil, oder wird wieder entfernt. Im folgenden sollen zuerst die Faserwerkstoffe näher betrachtet werden.

\subsection{Faserverbundwerkstoffe}
Faserverbundwerkstoffe werden durch Zusammenfügen mehrere Werkstoffe hergestellt, zum einen aus einer formgebenden Matrix und zum anderen aus den verstärkenden Fasern. Als Matrix werden häufig Epoxy- oder Polyesterharze verwendet, als Faserwerkstoffe Glas-, Kohlenstoff- oder Aramidfaser. Eine wichtige Eigenschaft der Faserverbundwerkstoffe ist die richtungsabhängige Festigkeit. Allgemein gilt, dass die Stabilität der Fasern in Längsrichtung um ein vielfaches höher ist als bei Scherkräften. Die Faserrichtung ist somit bei der Konstruktion und Fertigung auf jeden Fall zu beachten.  Die folgende Übersicht soll einen Überblick über die wichtigsten Vor- und Nachteile der verschiedenen Fasern und Harze bieten und so die Entscheidungsfindung je nach Anwendungsfall erleichtern.
\begin{itemize}
	\item Epoxydharz Vorteile
	\begin{itemize}
		\item Hohe statische und dynamische Festigkeit
		\item Geringer Härtungsschwund, gute Maßhaltigkeit
		\item Hohe Temperaturbeständigkeit
		\item Gute Chemikalien- und Witterungsbeständigkeit
		\item Geringe Brennbarkeit
	\end{itemize}
	\item Epoxydharz Nachteile
	\begin{itemize}
		\item Relativ teuer
		\item genaues Dosieren der Komponenten erforderlich
	\end{itemize}
	\item Polyesterharz Vorteile	
	\begin{itemize}
		\item Vergleichsweise günstig
		\item Anschleifen beim Überlaminieren meist nicht erforderlich
	\end{itemize} 
	\item Polyesterharz Nachteile
	\begin{itemize}
		\item Hoher Härtungsschwund
		\item Starker Styrolgeruch beim Verarbeiten
		\item Bei Verarbeitung ohne Sauerstoffausschluss leicht klebrige, riechende Oberfläche. Dadurch weniger Chemikalien- und Witterungsbeständig als Epoxydharz
	\end{itemize}	 
\end{itemize}

Zusammenfassend kann man sagen, dass Epoxydharz gegenüber Polyesterharz die besseren technischen Eigenschaften aufweist, dafür aber deutlich teurer ist. Da hier aber die Witterungsbeständigkeit und die geringere Schrumpfung wichtig sind wird Epoxydharz verwendet. Durch die geringere Schrumpfung ist in diesem Fall insbesondere ein leichteres Entformen zu erwarten.

Sowohl Epoxydharz- als auch Polyesterharzsysteme gibt es mit unterschiedlich Topf- bzw. Verarbeitungszeiten. Auch zu beachten ist, dass es Harze mit unterschiedlichen Anforderungen an die Bedingungen beim Härten gibt, insbesondere bzgl. Temperatur und Druck. Hier sollte ein passendes Produkt ausgewählt werden. In unserem Fall sind das Harze die bereits bei Raumtemperatur vollständig aushärten.

Im folgenden werden die häufig verwendete Kohle- und Glasfaser verglichen. Darüber hinaus gibt es noch weitere wie Aramid oder verschiedene Naturfasern.

\begin{itemize}
	\item Kohlefaser
	\begin{itemize}
		\item Hohe Zugfestigkeit
		\item Elektrisch leitend
		\item Gute Tränkungseigenschaften
		\item Vergleichsweise teuer
	\end{itemize}
	\item Glasfaser
	\begin{itemize}
		\item Etwas geringere Zugfestigkeit als Kohlefaser
		\item Elektrisch nicht leitend
		\item Teilweise schlechtere Tränkungseigenschaften
		\item Deutlich günstiger als Kohlefaser
		\item Bei gleicher Stabilität schwerer als Kohlefaser
	\end{itemize}
\end{itemize}

Die Entscheidung zwischen Kohlefaser und Glasfaser ist je nach Anwendung abzuwägen. Spielt das Gewicht eine eher geringe Rolle ist tendenziell Glasfaser aufgrund des günstigeren Preises zu bevorzugen.

%Quelle R&G_Handbuch.pdf

%Quelle: http://www.r-g.de/wiki/Was_sind_Faserverbundwerkstoffe%3F



\subsection{Wickelvorgang}

\subsubsection{Mathematik hinter dem Wickelvorgang}
Um die gewünschten Wickelmuster zu erzeugen ist einiges an Vorüberlegung notwendig gewesen. Das Hauptproblem besteht darin den Roving bei jedem Durchgang zielgenau mit einem definierten Versatz zum Vorherigen Durchgang auf das Rohr aufzubringen. Hierfür ist es wichtig den genauen Winkel der Spindelposition zu kennen sowie die exakte Position der vorherigen Wicklungen. Insbesondere die Enden an denen umgekehrt wird und entsprechend beschleunigt bzw. abgebremst wird können dabei zu Problemen führen. Um unterschiedliche Ansätze auszuprobieren wurden unter anderem Simulationen mit Matlab durchgeführt.

\subsection{Konstruieren in Siemens NX}
Der mechanische Teil der Wickelmaschine wurde in Siemens NX Konstruiert. Das endgültige Modell besteht aus mehreren Baugruppen die wiederum aus vielen einzelnen Modellen bestehen. Dabei existiert für jedes Einzelteil ein Modell das später mehrfach verwendet werden kann.  Mehrere Einzelteile die wiederum eine größere Einheit ergeben sind zu Baugruppen zusammengefasst. Auch diese lassen sich ähnlich wie die Modelle mehrfach verwenden. Auf diese Weise entsteht eine Hierarchische Struktur aus Baugruppen und Modellen die das Gesamtsystem ergeben. TODO Bilder von Modellen und Baugruppen einfügen

Während der Konstruktion können Variablen festgelegt werden die Bauteilübergreifend verwendbar sind. Die Ausrichtung innerhalb einer Baugruppe von den einzelnen Modelle zueinander wird mit sogenannten Zwangsbedingungen festgelegt. Eine Zwangsbedingungen kann z.B. festlegen, dass sich zwei Flächen berühren soll, oder Flächen bündig ausgerichtet werden sollen. 

Ein etwas spezielleres Feature das man nutzen kann sind sogenannte Wavelinks. Dabei können beispielsweise Konturen oder Kreise von einem Modell in ein anderes verknüpft werden und sind dort weiterverwendbar. Praktisch ist dies z.B. bei Bohrungslöcher für Schrauben, wobei die Position der Bohrungslöcher zueinander von einem anderen Modell übernommen wurde. Ist diese Verknüpfung assoziativ werden auch künftige Änderungen an dem Ursprungsmodell im verknüpften Modell übernommen.
    
\section{Konzept}
Der Grundaufbau der Wickelmaschine gleicht dem einer Drehbank. Es gibt eine Drehachse die den Wickelkörper rotiert, sowie einen Arm der auf einer Achse längs dem Wickelkörper bewegt wird. Dieser Arm führt den Roving und sorgt für eine exakte Platzierung, weshalb der Arm zusätzlich senkrecht zur Wickelachse bewegbar und rotierbar ist. Auf diese Weiße lässt sich der Abstand zwischen Wickelkörper und Arm varrieren sowie die Orientierung an den aktuellen Wickelwinkel anpassen. Zur Ansteuerung der Achsen kommen Schrittmotoren zum Einsatz, da diese zum einen vergleichsweise einfach und kostengünstig ansteuerbar sind und zum andern fertige Module mit Steppertreibern verfügbar sind. Über den oben beschriebenen drehbaren Arm wird der Roving zum Wickelkörper geführt, wobei dieser zuvor mit Epoxydharz getränkt werden muss, da dies je nach Gewebedicke nach dem Wickeln nur schlecht möglich ist. Außerdem ist durch die vorgesehene automatische Tränk- und Abstreifvorrichtung eine gleichmäßige Harzmenge garantiert. Durch eine variable Positionierung des Reitstocks sind Wickelkörper in beliebigen 

\section{Mechanik}
Der mechanische Aufbau der Wickelmaschine ist in Siemens NX modelliert. Manche Details wie Schrauben wurden dabei an irrelevanten Stellen weggelassen. Das Augenmerk liegt insbesondere auf den Teilen die in der Fräse selbst herzustellen sind, wie z.B. die Motorhalterungen, der Reitstock und die Verbindungen zu Teilen wie den Schrittmoten. Außerdem lässt sich in dem Maschinenmodell überprüfen, dass die Verbindungen und Lage der verschiedenen Teile zueinander stimmt. Im folgenden sollen die einzelnen Teile der Wickelmaschine näher betrachtet und erklärt werden.

\subsection{Grundplatte}
Die Grundplatte dient als Aufnahme für die weiteren Teile wie Spindel, Reitstock, Linearführunen usw. Aus Kosten- und Gewichtsgründen wird diese aus Holz hergestellt, wobei die Festigkeit aufgrund der geringen mechanischen Belastung ausreichend ist.

\subsection{Spindel}
Die Spindel nimmt die eine Seite des Wickelkörpers auf und treibt diesen mithilfe eines Schrittmotors an. Die Aufnahme für den Wickelkörper ist Kegelförmig um unterschiedlich Große Rohrdurchmesser von Innen Spannen zu können. Für spezielle Wickelkörper ist diese austauschbar.


\subsection{Reitstock}
Am Reitstock wird der Wickelkörper gegen gelagert, um unterschiedlich lange Wickelkörper verwenden zu können ist dieser frei positionierbar. Der Aufbau ist ähnlich dem von der Spindel mit Ausnahme des fehlenden Antriebs. Zusätzlich ist eine Feder vorgesehen die den Kegel auf den Wickelkörper drückt. Dies ermöglicht ein einfacheres wechseln des Wickelkörpers.  


\subsection{Wickelwagen}
Der Wickelwagen besteht aus dem Dreharm, usw...

\subsection{Dreharm}

\subsection{Tränkeinheit}



\section{Software}
Die verwendete Steuerung basiert auf [https://github.com/grbl/grbl grbl], welche bis zu vier Schrittmotoren anhand von G-Code ansteuert. Zum Übertragen bzw. streamen des G-Codes an die Steuerung existieren bereits mehrere fertige Programme, weshalb es nicht nötig ist hierfür ein neues zu entwickeln. Stattdessen geht es primäre darum G-Code zu erzeugen der zu dem gewünschten Wickelmuster führt. Hierfür wurde in C++ eine Software geschrieben die Anhand der eingestellten Parameter G-Code für die Maschine erzeugt. Anschließend wird dieser mit dem G-Code streamer an die Maschine übertragen und ausgeführt.

\end{document}